\subsection*{\Large Общая характеристика работы}
\fontsize{14pt}{15pt}\selectfont


\textbf{Актуальность темы исследования.}



\textbf{Целью диссертационной работы} является разработка модели образовательного ресурса и алгоритмов агрегирования и анализа образовательных данных в системе дистанционного обучения с использованием методов семантической паутины. 

Для достижения поставленной цели необходимо было решить следующие \textbf{задачи}:
\begin{enumerate}
 \item Разработать онтологические модели для описания учебных материалов, тестов, предметных областей учебных дисциплин, результатов и процесса обучения студентов, метрик оценки знаний и рейтинга студентов;
 \item Разработать методы автоматического агрегирования образовательных данных в системе дистанционного обучения с использованием методов естественной обработки языка и логического вывода и технологий Semantic Web и Linked Data;
 \item Разработать методы анализа качества и полноты образовательных материалов основанные на разборе семантических связей между объектами системы;
  \item Разработать новый подход к автоматизированному анализу знаний и рейтингов студентов системы дистанционного обучения с использованием статистических методов и семантических технологий;
  \item Создать программно-аппаратный комплекс системы дистанционного обучения на основе разработанных методов и проанализировать полученные результаты на практике.  
 \end{enumerate}

\textbf{Объектом исследования} является структура и логические отношения между объектами образовательного процесса, предметных областей учебных дисциплин и образовательных ресурсов.  

\textbf{Предметом исследования} является алгоритмическое обеспечение, предназначенное для автоматизации процессов наполнения системы дистанционного обучения учебными материалами, процессов анализа качества учебных материалов и знаний студентов. 

\textbf{Научная новизна.}
\begin{enumerate}
 \item Впервые ... . 
 \item Впервые ... .
 \item Впервые ... . 
\end{enumerate}

\textbf{Основные положения, выносимые на~защиту.}
\begin{enumerate}
 \item Модульная онтология, описывающая структуру образовательного процесса, структуру предметной обрасти учебной дисциплины, структуру тестов, действия и результаты студентов в системе дистанционного обучения.
 \item Методика наполнения образовательнной онтологии и агрегации учебных материалов в системе дистанционного обучения.
 \item Методика анализа качества и полноты учебного курса;
 \item Алгоритм анализа знаний и рейтинга студентов системы дистанционного обучения.
 \end{enumerate}



\textbf{Практическая значимость} диссертационной работы определяется ...

\textbf{Достоверность} изложенных в работе результатов обеспечивается ...

\textbf{Апробация работы.}
Основные результаты диссертационной работы докладывались и обсуждались на следующих конференциях:
International Conference on Knowledge Engineering and Semantic Web (Россия, Санкт-Петербург, 2013),
11th Extended Semantic Web Conference (Греция, Аниссарас, 2014), International Conference on Knowledge Engineering and Semantic Web (Россия, Казань, 2013), The 13th International Semantic Web Conference (Италия, Рива-дель-Гарда, 2014), 16th Conference of Open
Innovations Association FRUCT (Финляндия, Оулу, 2014).

Диссертационная работа была выполнена при поддержке грантов ...

\textbf{Личный вклад.} Автор принимал активное участие ...

\textbf{Публикации.} Основные результаты по теме диссертации изложены в ХХ печатных изданиях, Х из которых изданы в журналах, рекомендованных ВАК, ХХ --- в тезисах докладов.

%\underline{\textbf{Объем и структура работы.}} Диссертация состоит из~введения, четырех глав, заключения и~приложения. Полный объем диссертации \textbf{ХХХ}~страниц текста с~\textbf{ХХ}~рисунками и~5~таблицами. Список литературы содержит \textbf{ХХX}~наименование.

%\newpage
\subsection*{\Large Содержание работы}
Во \underline{\textbf{введении}} обосновывается актуальность исследований, проводимых в рамках данной диссертационной работы, приводится обзор научной литературы по изучаемой проблеме, формулируется цель, ставятся задачи работы, сформулированы научная новизна и практическая значимость представляемой работы.

\underline{\textbf{Первая глава}} посвящена ...

 картинку можно добавить так:
\begin{figure}[h] 
  \center
  \includegraphics [scale=0.27] {latex}
  \caption{Подпись к картинке.} 
  \label{img:latex}
\end{figure}

Формулы в строку без номера добавляются так:
$$
  \lambda_{T_s} = K_x\frac{d{x}}{d{T_s}}, \qquad
  \lambda_{q_s} = K_x\frac{d{x}}{d{q_s}},
$$

\underline{\textbf{Вторая глава}} посвящена исследованию 

\underline{\textbf{Третья глава}} посвящена исследованию 

В \underline{\textbf{четвертой главе}} приведено описание 

В \underline{\textbf{заключении}} приведены основные результаты работы, которые заключаются в следующем:
\begin{enumerate}
 \item Результат номер один.
 \item Результат номер два.
 \item Результат номер три.
% и так далее, если нужно
\end{enumerate}


%\newpage
\renewcommand{\refname}{\Large Публикации автора по теме диссертации}
\nocite{*}
\bibliography{biblio}